\begin{frame}\frametitle{What kind of patients are we talking about
anyway?}

\begin{itemize}[<+->]
\itemsep1pt\parskip0pt\parsep0pt
\item
  Children over the age of 1 year
\item
  Edema
\item
  Urine Protein:Creatinine ratio \textgreater{}= 2000mg/g
\item
  Urine Protein \textgreater{} 300mg/dL
\item
  Dipstick Urine protein 3+
\item
  Hypoalbuminemia (\textless{}= 2.5mg/L)
\end{itemize}

What is missing from the case definition?

\end{frame}

\begin{frame}\frametitle{Important resources to know}

\begin{itemize}[<+->]
\itemsep1pt\parskip0pt\parsep0pt
\item
  ISKDC - International Study of Kidney Disease in Children
\item
  KDIGO - Kidney Disease: Improving Global Outcomes
  (www.kdigo.org/home/glomerulonephritis-gn)
\end{itemize}

\end{frame}

\begin{frame}\frametitle{Obligatory Epidemiology Slide}

\begin{itemize}[<+->]
\itemsep1pt\parskip0pt\parsep0pt
\item
  1-3 (some reports as high as 7)/100,000 children under the age of 16
\item
  Black and Hispanic kids in the US more likely to have steroid
  resistant disease
\item
  Male to Female - from 2:1 to 3:2 in young children, equal in older
  kids (\textgreater{}8yo)
\item
  Lower incidence of steroid sensitive nephrotic syndrome in African
  children
\item
  Increased incidence (all types) in Asians (up to 6 times increase in
  some studies)
\end{itemize}

\end{frame}

\section{Steroid Sensitive Nephrotic Syndrome}

\begin{frame}\frametitle{Xiao Ma - The case}

Xiao Ma is a 3YO Asian male who presented to his local doc 3 days ago
with puffy eyes. The local doc gave cholorpheniramine and sent him home.
He comes back today with extension of the swelling to the feet and legs.

\end{frame}

\begin{frame}\frametitle{Xiao Ma - The discussion}

\begin{itemize}[<+->]
\itemsep1pt\parskip0pt\parsep0pt
\item
  Most likely what time of day did he present initially?
\item
  What tests do you want to do?
\item
  What therapy should you start?
\item
  What is the most important predictor of outcome in Xiao Ma's case?
\end{itemize}

\end{frame}

\begin{frame}\frametitle{Initial approach to therapy for Childhood
Nephrotic Syndrome}

\begin{itemize}[<+->]
\itemsep1pt\parskip0pt\parsep0pt
\item
  Steroids are the mainstay
\item
  Initial dose is 2mg/kg/day or 60mg/m2/day in single daily dose
\item
  Don't reduce the dose for at least 4 weeks, better to go for 6 weeks
\item
  Follow up dose of 1.5mg/kg alternate days and tapered over 2 - 5
  months
\end{itemize}

\end{frame}

\begin{frame}\frametitle{Why so long?}

Hodson, et.al. did some meta-analysis of RCTs using steroid therapy
regimens.

\begin{longtable}[c]{@{}lll@{}}
\hline\noalign{\medskip}
Objective & Result & Stats stuff
\\\noalign{\medskip}
\hline\noalign{\medskip}
3 vs.~2 months & 30\% relapse reduction & RR 0.7 (0.58-.84)
\\\noalign{\medskip}
6 vs.~3 months & reduction in 12-24m relapse & RR 0.57 (0.45-0.71)
\\\noalign{\medskip}
\hline
\end{longtable}

\end{frame}

\begin{frame}\frametitle{Relapse Therapy - The Return of Xiao Ma}

Poor Xiao Ma got a cold. It has been 5 months since his original episode
but now he has three plus protein in his urine by mom's home albustix.
She calls the office for advice.

\begin{itemize}[<+->]
\itemsep1pt\parskip0pt\parsep0pt
\item
  What are you going to tell her?
\end{itemize}

\end{frame}

\begin{frame}\frametitle{Approach to Relapse Therapy in Childhood
Nephrotic Syndrome}

\begin{itemize}[<+->]
\itemsep1pt\parskip0pt\parsep0pt
\item
  Relapses happen - infrequent is better than frequent
\item
  Prednisone dose is the same initially, treat until protein free for 3
  days (trace or less)
\item
  After initial therapy give 1.5mg/kg every other day for 4 weeks
  minimum
\item
  Infrequent relapses = 3 or fewer per year.
\end{itemize}

\end{frame}

\section{Frequent Relapse/Steroid Dependent Therapy}

\begin{frame}\frametitle{Xiao Li}

\end{frame}

\section{Steroid Resistant Nephrotic Syndrome}
